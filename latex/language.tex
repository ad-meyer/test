%%%%%%%%%%%%%%%%%%%
% Sprachcodierung %
%%%%%%%%%%%%%%%%%%%
\usepackage[T1]{fontenc}
% Paket fontenc: Fontencoding T1 festlegen, damit Wörter, die Umlaute
% enthalten richtig getrennt werden.
\usepackage[utf8]{inputenc}
% Paket inputenc: Unterstützung verschiedener Zeichenkodierungen.  Unterstützte
% Kodierungen sind:
% ascii		ASCII Kodierung für Zeichen in dem Bereich von 32 bis 127
% latin1	ISO Latin1
% latin2	ISO Latin2
% latin3	ISO Latin3
% latin4	ISO Latin4
% latin5	ISO Latin5
% latin9	ISO Latin9 (definiert die ISO 8859-15 Codierung einschließlich des EURO-Zeichnes: \texteuro)
% decmulti	DEC Multinational Character Set Encoding für das OpenVMS Betriebssystem
% cp437		IBM 437 Code Page
% cp437de	IBM 437 Code Page deutsche Version
% cp850		IBM 850 Code Page
% cp852		IBM 852 Code Page
% cp865		IBM 850 Code Page
% cp1250	Windows 1250 (Zentral- und Osteuropa) Code Page
% cp1252	Synonym für ansinew
% applemac	Macintosh Kodierung
% next		Next Kodierung
% ansinew	Windows 3.1 ANSI Kodierung, erweiterung von Latin1
% Dokumentation zu diesem Paket: /usr/share/doc/texmf/latex/base/inputenc.dvi.ps
%
\usepackage[english,ngerman]{babel}
% Paket babel: Unterstützung für andere Sprachen als amerikanisches Englisch
% Die Sprachen Englisch, Französich und Deutsch  verwenden. Die zuletzt aufgeführte
% Sprache ist die Sprache, die beim Start geladen wird. Zwischen den einzelnen
% Sprachen kann man mittels \selectlanguage{english}, \selectlanguage{ngerman} und
% \selectlanguage{french} umgeschaltet werden. ngerman steht für die neue, german
% für die alte deutsche Rechtschreibung.
% Mittels \iflanguage{sprache}{wahr-Klausel}{falsch-Klausel} kann abgeprüft werden,
% welche Sprache gerade aktiv ist und demensprechend können Befehle ausgeführt werden.
% Umlaute und Sonderzeichen mit dem babel Paket:
% ä = \"a 
% Ä = \"A
% ö = \"o 
% Ö = \"O
% ü = \"u 
% Ü = \"U
% ß = \"s
% SS = \"S ( groß geschriebene ß als SS wie z.B in STRASSE )
% Deutsche doppelte Anführungszeichen links unten \glqq
% Deutsche doppelte Anführungszeichen rechts oben \grqq
% Deutsche einfache Anführungszeichen links unten \glq
% Deutsche einfache Anführungszeichen rechts      \grq
% Französische doppelte Anführungszeichen links  \flqq
% Französische doppelte Anführungszeichen rechts \frqq
% Franz<F6>sische einfache Anf<FC>hrungszeichen links  \flq
% Franz<F6>sische einfache Anf<FC>hrungszeichen rechts \frq
% Original Anf<FC>hrungszeichen \dq
% Weitere Spezialbefehle zur Silbentrennung und Lignatur
% Dokumentation zu diesem Paket: /usr/share/doc/texmf/generic/babel/user.dvi.gz
\selectlanguage{ngerman}
