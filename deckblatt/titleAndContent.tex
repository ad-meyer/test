\pagestyle{empty}
%
% Deckblatt
%
\begin{figure}[p]
\begin{tikzpicture}
	% Rahmen
	\draw[rounded corners=4mm] (151mm,213mm) -- (151mm,196mm) -- (0mm,196mm) -- (0mm,145mm) -- (151mm,145mm) -- (151mm,63mm);
	% F
	\draw (62mm,3mm) -- (62mm,59mm) -- (102mm,59mm) (62mm,31mm) -- (83mm,31mm);
	% A
	\draw (73mm,3mm) -- (106mm,58mm) -- (106mm,3mm) (90mm,31mm) -- (106mm,31mm);
	% U
	\draw (113mm,58mm) -- (113mm,18mm) arc (180:360:19mm)  -- (151mm,58mm);

	% FAU Logo
	\node at (25mm,175mm){\includegraphics[keepaspectratio=true,width=35mm]{deckblatt/FAU-Siegel.png}};
	% Arbeitsart
	\node at (25mm,151mm){\Large\thethesis};
	% Titel
	\node at (97mm,175mm){
		\parbox{10cm}{
			\centering \bfseries \itshape \Large \thesisTitleFormatted
		}
	};
	% Autor
	\node at (97mm,151mm){
			\bfseries \itshape \Large \myname
	};
	
	\node at (60mm,59mm)[below left,inner sep=0]{
		\parbox{6cm}{
			\begin{flushright}
				\large Lehrstuhl für Informatik 6 \\ (Datenmanagement)
			\end{flushright}
		}
	};
	\node at (60mm,25mm)[above left]{
		\parbox{6cm}{
			\begin{flushright}
				\large Department Informatik
			\end{flushright}
		}
	};
	\node at (60mm,-2mm)[above left,inner sep=0]{
		\parbox{6cm}{
			\begin{flushright}
				\large Friedrich Alexander-\\Universität\\Erlangen-Nürnberg
			\end{flushright}
		}
	};
\end{tikzpicture}
\end{figure}
%
% Titelseite
%
\begin{titlepage}

  \begin{center}
    
    {\Huge \bfseries
      \thesistitle{}\\
    } 
    
    \vspace*{1cm}
    \thethesis{} im Fach Informatik
    \vspace{2cm}
    
    {\large vorgelegt von} \\
    \vspace*{0.7cm}
    {\Large \bfseries \myname} \\
    \vspace*{0.7cm}
    {\large geb. \birthday{} in \birthplace{}} 
    
    \vspace{1.5cm}
    
    angefertigt am 

    \vspace{1cm}
    
    {\bfseries
      Lehrstuhl für Informatik 6 (Datenmanagement) \\
      Department Informatik \\
      Friedrich-Alexander-Universität Erlangen-Nürnberg (FAU)\\
      }
    
    \vspace{0.5cm}
\end{center}
\begin{tabbing}
    Betreuer: \= \corrector{}\\
    \> \tutor{} 
\end{tabbing}
    \vspace{0.25cm}
    
\begin{tabbing}
  Beginn der Arbeit: \startofwork{} \\
  Abgabe der Arbeit: \dayofdoom{}
\end{tabbing}

\end{titlepage}
%
% Leere Seite für doppelseitiges Layout einfügen
\cleardoubleemptypage
%
% Leeres Seitenlayout: Keine Seitennummern, kein Header oder Footer
%
Ich versichere, dass ich die Arbeit ohne fremde Hilfe und ohne Benutzung
anderer als der angegebenen Quellen angefertigt habe und dass diese Arbeit in
gleicher oder ähnlicher Form noch keiner anderen Prüfungsbehörde
vorgelegen hat und von dieser als Teil einer Prüfungsleistung angenommen
wurde. Alle Ausführungen, die wörtlich oder sinngemäß übernommen
wurden, sind als solche gekennzeichnet.

\vspace{2cm}

\noindent
Der Friedrich-Alexander-Universität Erlangen-Nürnberg (FAU), vertreten durch
den Lehrstuhl für Informatik 6 (Datenmanagement), wird für Zwecke der
Forschung und Lehre ein einfaches, kostenloses, zeitlich und örtlich
unbeschränktes Nutzungsrecht an den Arbeitsergebnissen der \thethesis{}
einschließlich etwaiger Schutzrechte und Urheberrechte eingeräumt.

\vspace{2cm}
Erlangen, den \dayofdoom{}

\vspace{2cm}
\myname{} \hfill \ 

\vspace{0,5cm}
%
% Leere Seite für doppelseitiges empty Layout einfügen
\cleardoubleemptypage
%
% Seitenlayout mit Seitennummern, allerdings kein Header oder Footer
\pagestyle{plain}
%
% Römische Seitenzahlen
\pagenumbering{roman}
% Da wir 6 Seiten ohne Seitennummerierung im Seitenlayout empty verwenden, reicht ein 
% \usepackage[pdftex,pdfpagelabels,plainpages=false,]{hyperref} nicht mehr aus. Es muss zusätzlich noch der 
% Seitenzähler auf 7 anstatt wie defaultmäßig auf 1 für die folgend Seiten gesetzt werden.
\setcounter{page}{7}
%
\chapter*{Abstract}
\section*{\thesistitleGB}
\selectlanguage{english}
TODO Abstract
 
\clearpage{\pagestyle{empty}\cleardoublepage}
\chapter*{Kurzfassung}
\section*{\thesistitle}
\selectlanguage{ngerman}
TODO Kurzfassung

%
% Leere Seite für doppelseitiges plain Layout einfügen
\cleardoubleplainpage
%
% Inhaltsverzeichnis
\tableofcontents
%
% Leere Seite für doppelseitiges plain Layout einfügen
\cleardoubleplainpage
%
% Abbildungsverzeichnis
\listoffigures
%
% Leere Seite für doppelseitiges plain Layout einfügen
\cleardoubleplainpage
%
% Tabellenverzeichnis
\listoftables
%
% Leere Seite für doppelseitiges plain Layout einfügen
\cleardoubleplainpage
%
% Listingverzeichnis
\lstlistoflistings
%
% Leere Seite für doppelseitiges plain Layout einfügen
\cleardoubleplainpage
%
% Seitenlayout mit Seitennummern, Header und Footer
\pagestyle{headings}
%
% Arabische Seitenzahlen
\pagenumbering{arabic}
%
% Seitenzähler auf 1 setzen, das benötigt das Paket hyperref
\setcounter{page}{1}
